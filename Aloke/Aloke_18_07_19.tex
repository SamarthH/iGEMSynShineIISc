\documentclass [a4paper,11pt]{article}
\usepackage[margin=0.5in]{geometry}
\usepackage{graphicx}
\usepackage{physics}
\usepackage{mathrsfs}
\usepackage{amsmath}
\usepackage{amssymb}

\newcommand{\R}{\mathbb{R}}
\newcommand{\F}{\mathbb{F}}
\begin{document}
\title{Discussions with Prof. Aloke Kumar}
\author{Praveen J and Samarth Hawaldar}
\date{$18^{th}$ July 2019}
\maketitle
\textbf{Subject: Validation of methods and brain-storming for new methods}\\
In order to validate methods of bacterial population we had brain stormed earlier, we mailed and met Prof. Aloke Kumar, mechanical department, IISc. He works in fields which include bacteria, microfluidics, opto-electrofluidics and biophysics. We presented our project and our ideas first, which included using impedance or viscosity to determine population, fluorescence tagging and scattering and few other cytometric methods and OD determination. He was quite apprehensive about the impedance method and also mentioned that the viscosity method might not be reliable. He suggested the method of using zeta potential to determine the ratio of bacteria. Using the signal strength of different Fourier components of the scattered light, if it is bi-modal (the zeta potential of the two are different such that the peaks are non-coincidental), we can determine the ratio of populations. He suggested testing the above using the zeta measuring device in the CENSE department of IISc and cross verify the populations by performing flow cytometry of the same samples. He mentioned that one of this student had used the zeta meter recently and guided us through the process of using the same. He also mentioned suitable voltage ranges and electrodes to use when we being to construct the device.\\
He also mentioned that our project was quite new and he hadn't heard of any such before and that he found it interesting. He wished us the very best and asked to keep him updated of our progress.
\end{document}