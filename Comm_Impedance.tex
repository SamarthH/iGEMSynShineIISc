\documentclass [a4paper,11pt]{article}
\usepackage[margin=0.5in]{geometry}
\usepackage{graphicx}
\usepackage{physics}
\usepackage{mathrsfs}
\usepackage{amsmath}
\usepackage{amssymb}

\newcommand{\R}{\mathbb{R}}
\newcommand{\F}{\mathbb{F}}
\begin{document}
\title{Communication Protocol for Testing Impedance Microbiology}
\author{Synshine Engineering Team}
\date{}
\maketitle

Using the circuit provided, we will be scanning in the range from f-value closest to 30 Hz to 30 kHz. The increments are to be done in a factor of two. The circuit will be tested for 14 different frequencies.

The following things are black-boxed:
\begin{itemize}
\item int init\_adc() : Initializes ADC. If it worked, then returns 0, else -1.
\item int init\_sine() : Initializes the sine wave generator.If it worked, then returns 0, else -1.
\item long read\_val\_adc(int a) : This function does the following :
	\begin{itemize}
		\item If `a' is any one of 0,1,2, or 3, it returns the 24-bit value of the $a^{th}$ ADC as a long .
		\item If `a' is not one of the above the function returns `-1'
	\end{itemize}
\item int set\_val(long freq) : This sets the closest frequency possible in the sine wave generator.If it worked, then return 0, else -1.
\item int change\_freq\_steps(long n\_steps) : Adds $0001_{hex} \times n\_steps$ to the output to the sine-wave generator. If it overflows, then return -1.
\item int mult\_freq(long k) : Multiplies the output to ADC by k. If it overflows, then return -1.
\item int set\_sine(int a) : If `a' is 1, turn on the sine wave generator else turn it off.
\end{itemize}

The arduino responds to the command of the computer through serial communication.
\section{Transmission Protocol for the Primary message}
The primary message will be the initial message before validation procedures.
\subsection{From Computer to Arduino}
The computer can send the following :
\begin{itemize}
\item SETFR:\# = This commands the arduino to set the frequency to the value closest to \# which is a long
\item CGSTP:\# = This commands the arduino to change the frequency steps by \# which is a long
\item MLSTP:\# = This commands the arduino to multiply the frequency by \# which is a long
\item CHKCF: = This commands the arduino to check the current frequency
\item GENHI: = This commands the arduino to turn on the sine wave generator
\item GENLO: = This commands the arduino to turn off the sine wave generator
\item ERROR:$<>$ = This sends an error to the arduino
\end{itemize}
\subsection{From arduino to computer}
The arduino can send the following:
\begin{itemize}
\item ERROR:$<>$ = This sends an error to the computer
\item SDDAT:\#:c =  This sends the data to the computer as \# = ADC output of the $c^{th}$ ADC
\end{itemize}
\end{document}